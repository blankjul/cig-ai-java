\section{Introduction} 
\label{sec:intro}


Due to the fact that the game industry was growing over the past years~\cite{gartner}
constructing good game artificial intelligences \ac{AI} comes more and more important.
Normally a game \ac{AI} is created for one specific game. Often if humans play against
enemies there are different levels of difficulties that could adjusted for example easy, middle and hard.
Our aim of the course "Computational Intelligence in Games"~\footnote{http://is.cs.ovgu.de/Courses/Team+Projects/Computational+Intelligence+in+Games.html} at the Otto-von-Guericke-University Magdeburg in Germany~\footnote{http://www.ovgu.de/} was
to evaluate three different approaches on the General Video Game AI Competition~\footnote{http://www.gvgai.net/}.
The game AI should play unknown games as effectively as possible. The agent can observe the
whole grid with game objects that are moving. Besides he gets information of all the collision that happened.
Moreover the agent has 40 ms for each game step to act. All actions - depending on the game - are  LEFT, UP, RIGHT, DOWN, USE and NIL.
To find the best action the agent could simulate steps and so predict the next state. But there is uncertainty because the game objects move randomly.

All this facts form a complex problem to solve. The agent should play tight in order to win most of the games as fast as possible. 
Before each of the games is starting the aim of the controller is unknown. Therefore he should dynamically find out what is the target of the current game.

In this paper we introduce three general approaches for a game \ac{AI}. Therefore we first outline our literature review and explain after that
the background. The theory could not always be implemented like it was proposed because it has to
fit the requirements of the games. For this reason we explain at the chapter 4 the details of our implementations.
Then we evaluate first the best parameters of each approach and thereafter find out which agent is our winner.








