\section{Literature Review} 
\label{sec:lit}

The field of \ac{CIG} comprises a very large area of research and there is a lot of work relating to this topic. We are dealing with \ac{GVGP}~\cite{BatesCongdon2013}, which belongs to \ac{CIG}. 
An essential work from T. Schaul~\cite{schaul2013pyvgdl} depicts how our game environment is built up. A \ac{GVGL} is presented, in which games can be described in a text file without the knowledge of game programming. Different objects can be created and rules can be defined deciding what happens if one object collides with another. It was created for the research of \ac{AI} in games and creating very fast prototypes of General Video Games. The agent who plays the game gets information about the actual state and can chose an available action. In our case, the agent gets a global observation grid and not only a local one (e.g. first-person view). 

There are several conferences dealing with the topic of \ac{CIG} auch as the IEEE Conference on \ac{CIG} or the AAAI Conference on Artificial Intelligence and Interactive Digital Entertainment (AIIDE). The field of \ac{CIG} contains a lot of different research areas beside \ac{GVGP} and agent decision making. Yannakakis, G.N. and Togelius, J. pointed out 10 main research areas of \ac{CIG}: "NPC behavior learning, search and planning, player modeling, games as AI benchmarks, procedural content generation, computational narrative, believable agents, AI-assisted game design, general game artificial intelligence and AI in commercial games"~\cite{panorama}.
This shows the variety of its research field. The subject of \ac{GVGP} often makes use of \ac{MCTS}. The comprehensive description of methods~\cite{BrowneMCTSurvey} describes nearly all approaches for \ac{GVGP} using \ac{MCTS}. We used this approach to cover the domain of \ac{RL} based algorithms.