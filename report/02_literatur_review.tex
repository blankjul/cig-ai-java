\section{Literature Review} \label{sec:lit}
\todo{
Perform a literature review: What has been done before in this field? What is the main technique/s used in the paper, and what has it/they been used for in the literature before? Give references to the most relevant work published. Example: ~\cite{BrowneMCTSurvey}.
}

The field of \ac{CIG} is very wide and there is a lot of work related  to this theme. We are dealing with \ac{GVGP}, which belongs to the topic \ac{CIG}. 
An essential work from T. Schaul~\cite{schaul2013pyvgdl} shows how our game-environment is build-up. He describes a general video game language in which games can be described in an easy way without the knowledge of programming games. Different objects and be created and rules can be defined which decide what happens if one object collides with another. It was created for research in \ac{AI} in games. The agent which plays the game gets information (observation) about the actual state and can chose an available action. In our case, the agent gets a global observation about the game and not only a local one (e.g. first-person view). The games in this competition were created with this language to have a good basis for our algortihms.
There several are conferences which deal with the topic of \ac{CIG} e.g. the IEEE Conference on Computational Intelligence and Games (CIG) or the AAAI Conference on Artificial Intelligence and Interactive Digital Entertainment (AIIDE). The field of \ac{CIG} is very wide and contains a lot of different research areas beside \ac{GVGP} and agent decision making. Ashlock and McGuinness for example showed methods for \ac{PCG}