\section{Conclusions and Future Work} 
\label{sec:conc}

This paper presents a comparison of several approach to play general video games. We selected for each research area (\ac{HR}, \ac{RL}, \ac{NI})
one algorithm and implemented that. 


For the evaluation we forced to look at the data for example average wins, score and time steps.
We ignored the in-game behaviour of our agents completely and only looked at the statistics. 
Furthermore we were limited by time and computational power which restricts the iterations of our experiment. 
To get more significant results experimental setup with more than 1000 iterations would stand for an higher validity.
The \ac{HR} approach has the best results (depending on the number of wins) and was better than the \ac{NI} and \ac{MCTS}. This might have serveral reasons. Firstly our implementation of \ac{MCTS} and \ac{NI} is maybe not very effective or we did not find the optimal parameters to fit our problem. We can not make statements about the accuracy of the approaches (\acs{HR}, \acs{RL}, \acs{NI}) using our conttrollers as quality criterion.
For futue work the implementation of other controllers and methods (e.g. Neuronal Nets, Pheromone) to compare them with the the actual ones would be an idea. Furthermore a test with unknown games would be good, to compare the agents and the methods on completely unknown games.

-combine approaches 
-parameter systematisch druchgehen -> keine powwer
-heuristic ist besser .> EA und MCTS finden nix wenn es nur wenig sieg sprites gibt
-

\todo{
Explain the main contributions of this work: what are the most important findings. Finally, explain how could this work be extended. What would be the next steps?
}

